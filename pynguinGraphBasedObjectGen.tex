\documentclass[%
  chapterprefix=false,%
  open=right,%
  twoside=true,%
  paper=a4,%
  logofile={Figures/logo.png},%
  thesistype=master,%
  UKenglish,%
]{se2thesis}
\listfiles
\usepackage[ngerman,main=UKenglish]{babel}
\usepackage{blindtext}
\usepackage[%
  csquotes=true,%
  booktabs=true,%
  siunitx=true,%
  minted=true,%
  selnolig=true,%
  widowcontrol=false,%
  microtype=true,%
  biblatex=true,%
  cleveref=true,%
]{se2packages}

\usepackage{algorithm}
\usepackage{algpseudocode}
\usepackage{multirow}
\usepackage{amsmath}
\usepackage{hyperref}
\usepackage[caption=false]{subfig}

\addbibresource{ref.bib}



\newcommand{\classname}[1]{\texttt{#1}}
\newcommand{\callable}[2][]{\(\text{\texttt{#2}}(#1)\)}
\newcommand{\field}[1]{\texttt{#1}}

\author{Gonzalo A. Oberreuter Álvarez}
\title{Effects of the Implementation of a Graph-Based Object Synthesis Heuristic \\ on Pynguin}
\degreeprogramme{Computer Science}
\matrnumber{110082}
\supervisor{Prof.\,Dr.~Gordon Fraser}
\external{Prof.\,Dr.~Christian Hammer}
\advisor{}
\department{Faculty of Mathematics and Informatics}
\institute{Chair of Software Engineering}
\location{Passau}

\begin{document}

\frontmatter

\maketitle

\iffalse{}

\authorshipDeclaration{}

\begin{abstract}
  An English abstract to the thesis. 
  TBD.\@
\end{abstract}

\begin{abstract}[german]
  Eine deutschsprachige Zusammenfassung der Arbeit.
  TBD.\@
\end{abstract}

\begin{acknowledgements}
  Some acknowledgements. 
  TBD.\@
\end{acknowledgements}
\fi

\tableofcontents



\mainmatter{}

\chapter{Introduction}\label{chap:introduction}

Software testing is one of the key aspects of software development while trying to ensure quality over a final product, regardless of the context in which the developing process is made.
This quality can be achieved by the insight provided by the result of the tests, and even because of the defects that can be encountered during the testing phase.
Nonetheless, even though coding different kinds of tests is a good practice, it is often ignored by new or inexperienced developers, who also make this mistake halfway by not getting a complete introspection of their own code or, in other words, not getting a complete kind of test coverage.
As of 2017, a study by Trauch and Grabowski~\cite{DBLP:conf/icst/TrautschG17} presented that, over more than 4 million tests, most of them were not correctly categorized (as unit test or not) and approximately half of them use mocking as a testing technique, which tells some aspects about the developing community of the projects in review. 
With this general idea into mind, is that researchers in the last decade   have put effort into autonomous test generation, the concept that implies the usage of different methods or techniques in order to identify patterns and generate test sets with little to zero external intervention.
In 2015, empirical proof was found by Fraser et al.~\cite{DBLP:journals/tosem/FraserSMAP15}, who showed that the usage of automated Java unit test generation increases the general structural coverage, does not lead to the detection of more faults and affects negatively the ability to capture intended class behaviour.
The results of this study state fundamentally that this behaviour  comes from the early stages of the tool in question, and proposes to put more work into the readability of generated tests and  the process of test making itself.

Within the scope of Python testing, Pynguin~\cite{DBLP:conf/icse/LukasczykF22} is command line interface for the generation of unit tests that applies various algorithms for input generation, such as DynaMOSA~\cite{DBLP:journals/tse/PanichellaKT18}, MIO~\cite{DBLP:conf/ssbse/Arcuri17}, MOSA~\cite{DBLP:conf/icst/PanichellaKT15}, Whole Test Suite~\cite{DBLP:journals/tse/FraserA13}, among others.

At the time of its original release (25th of July 2020), Pynguin was a state-of-the-art open source tool that has been the first step for subsequent researches about how to improve the performance and functionalities of Pynguin itself, including CodaMOSA~\cite{DBLP:conf/icse/LemieuxILS23}, PyLC~\cite{DBLP:conf/sac/SalariEAS23} and many other tools that are briefly explained in the Section~\ref{chap:related_work}.
These recent new extensions of Pynguin and the latter aforementioned problem at the time of generating complex object inputs are the principal motivations of this thesis and the related and future work to it.

The original Pynguin research paper~\cite{DBLP:conf/icse/LukasczykF22} stated that after the test generation of 118 Python modules, the average branch coverage over all algorithms was $66.8\%$, which leads to think that improvement is possible.

Although automatic test generation is achievable while having type information available for statically typed languages like Java~\cite{DBLP:journals/tse/FraserA13} or C, dynamically typed languages such as Python, JavaScript or Lua enforce further problems at the time of generating correct arguments for the execution of methods or functions under test.
The major concerns about the argument generation, is that the lack of type information produces ambiguity for the heuristics of the tool at hand at the moment of synthesizing non-primitive types.
Also, the complexity of these class instances in terms of their internal field values might produce runtime errors, which imply a local optima in the search landscape of the test representation~\cite{DBLP:conf/sigsoft/0001O00D21} and therefore an upper bound for coverage.
This latter idea, brings a recurrent problem in the automatic test generation for languages that include any kind of Object-Oriented Programming, which is the reaching of branches that need an object instance as input in a specific variable-state.

\begin{figure}
  \inputminted[linenos]{python}{Figures/example.py}
  \caption{Module example.py\label{lst:1}}
\end{figure}

\begin{figure}
\inputminted[linenos]{python}{Figures/dependencies.py}
\caption{Dependency module of example.py\label{lst:2}}
\end{figure}

\begin{figure}
  \inputminted[linenos]{python}{Figures/test1.py}
  \caption{Test suite generated by Pynguin for module example.py\label{lst:3}}
\end{figure}

As an example, Listings~\ref{lst:1}~and~\ref{lst:2} represent a Module Under Test (MUT) and its dependency module, respectively, in which lines 8, 10 and 12 of the MUT are the targets of the branch coverage problem.
Even though the test suite generated by Pynguin (see Listing~\ref{lst:3}) gets to a branch coverage of \(37.5\%\) (using DynaMOSA, the seed 1998 and 100 seconds as stopping condition), it can be inferred that this result comes from \verb|test_case_1()|, that executes one of the first two branches in the MUT and checks if the variables \verb|actor| and \verb|target| are actual \verb|Player| objects.
This behaviour prevents Pynguin from escaping a local optima for the branch coverage of line 12.

The current Master's thesis proposes an addition to the original structure of Pynguin, specifically to the \classname{GenerationAlgorithm} abstract class, by developing a Graph-Based Object Synthesis approach~\cite{DBLP:conf/sigsoft/0001O00D21} (GBOS from now on) for the static analysis of Python at bytecode level in order to generate suitable object inputs, diminish the branch coverage gap, and empirically study the effect of the presence of type information in this same generation heuristic.

Coming back at Listing~\ref{lst:1}, the idea of using a graph for the synthesis of objects is to get Pynguin to generate a \verb|GameState| object with the necessary \verb|Players| in the \verb|players| list, and an \verb|Action| object with correct attributes, so a hypothetical test could reach the third branch in the MUT, even if it is with a false guard.

The Graph Based heuristic proposed by Lin et al.~\cite{DBLP:conf/sigsoft/0001O00D21} firstly generates an Object Construction Graph (OCG), which is a sub-graph of a previously generated Program Dependency Graph (PDG).
Then, this OCG is used to generate a code template that should be translated directly into the test code representation of Pynguin, for it to be then evolved or modified by one of the available Search-Based algorithms.
Pynguin, similarly to other Search-Based test generating tools, represents its test cases as a sequence of implementations of a super class \classname{Statement} that can be later transformed into an Abstract Syntax Tree (AST) and an actual block of Python code.
At its time, the GBOS heuristic was completely implemented in and for Java, which means that part of the work done was to ideate a Python representations of the OCG.\@
Listing~\ref{lst:4} shows the code of a hypothetical statement sequence template for the testing of the example module, that still gets a run time error in line 18.
However, this template allows Pynguin to get out of the local optima thought mutations, if right values are modified in the correct variables.

\begin{figure}
  \inputminted[linenos]{python}{Figures/template.py}
  \caption{Potential test template obtained through the use of an OCG\label{lst:4}}
\end{figure}

A type information gathering mechanism (e.g.~the analysis of every method or function's AST) is applied throughout the study in order to review the modules in the testing set of the experimental setup on how the general type information of a Python module correlates with the ability of the GBOS approach to generate correct object inputs templates and its results.
The reason for this is, as stated before, Python being a dynamically typed language and not requiring the type of variables in the script in order to be executed.
To illustrate this, if line 4 in Listing~\ref{lst:1} were to be replaced to \mint{python}|def checkRules(self, action, state) -> bool:| it would make Pynguin not have any information about the type of the parameters, and generate a test suite similar to the one in Listing\ref{lst:5}.
The behaviour that allows Pynguin to generate the previous suite imply that the current type system implemented as part of it does not work completely well when trying to infer object types.
Therefore, this idea tries to measure the coverage result effect of the variable amount of type information available.

\begin{figure}
  \inputminted[linenos]{python}{Figures/test2.py}
  \caption{Test suite generated by Pynguin for a variation of module example.py\label{lst:5}}
\end{figure}

After the implementation part was done, a study of the extension was made through the selection of arbitrary sets of Python modules, being these a combination of the ones used for various studies done to Pynguin's updates over time.

As a summary, the main contributions of this study are

\begin{itemize}
  \item An implementation of the GBOS heuristic described by Lin et al.~\cite{DBLP:conf/sigsoft/0001O00D21} as an extension to the automatic test generation tool for Python, Pynguin.
  \item An empirical analysis over the effects of the usage of the GBOS heuristic, when run on a set of Python modules that aim to generalize the current state of publicly available modules.
  \item The discussion about the different implications that come when using the GBOS extension, and the possible future improvements to it.
\end{itemize}

The Pynguin extension can be found, together with the example module presented in the current section, in the same GitLab repository of Pynguin, under the branch named ``GBOS''.

\chapter{Background}\label{chap:background}

\section{Automatic Test Generation}

\newpage

\section{Program Slicing}\label{sec:slicing}

While being part of the studied heuristic, program slicing is the technique of obtaining a subset of program statements according to a point of interest inside the program at hand.
Within the scope of the GBOS heuristic, the type of slicing that bring the most interest is static slicing, which is done merely with the information that can be obtained statically from a program.
Although slicing is a quite versatile tool, it is mostly used and related to the Program Dependence Graph (PDG), structure that represents all dependencies between statements or blocks in a program.

A Control Flow Graph (CFG), is the first step to the gathering of data flow inside a program, as it defines the explicit jumps between statements.
It can be defined as \(\text{CFG} = \langle V_{\text{CFG}}, E_{\text{CFG}} \rangle\) where every node represents an abstraction of statements or basic blocks and every directed edge, a path of execution.

The Control Dependence Graph (CDG) is the second relevant structure working as a requisite for slicing.
Before defining the elements of a CDG, it is necessary to define the concepts
\begin{itemize}
  \item \textbf{Dominator}: a statement \(v_1\) dominates another statement \(v_2\), if every execution path to \(v_2\) in the CFG goes through \(v_1\).
  \item \textbf{Post-Dominator}: a statement \(v_1\) post-dominates another statement \(v_2\), if every execution path from \(v_2\) to an exit node in the CFG goes through \(v_1\).
\end{itemize}
Then, a \(\text{CDG} = \langle V_{\text{CDG}}, E_{\text{CDG}} \rangle\) is a graph where \(V_{\text{CDG}} = V_{\text{CFG}}\), and for every pair \(v_1, v_2 \in V_{\text{CDG}}\), there is a directed edge \((v_1, v_2)\) if and only if \(v_2\) is not a post-dominator of \(v_1\) and there is a path in the \(CFG\) between \(v_1\) and \(v_2\) whose nodes are not post-dominated by \(v_2\).
This edge represents a control dependency between both statement blocks.

Similarly, the data dependencies of a program can be represented with a Data Dependence Graph \(\text{DDG} = \langle V_{\text{DDG}}, E_{\text{DDG}} \rangle\), where \(V_{\text{DDG}} = V_{\text{CFG}}\) and for every pair \(v_1, v_2 \in V_{\text{DDG}}\), there is a directed edge \((v_1, v_2)\) if and only if there is a variable \(X_{v_1}\) defined in \(v_1\) and used in \(v_2\), that is part of the reaching definitions of \(v_2\).

Reaching definitions~\cite{DBLP:books/aw/AhoSU86} is a data-flow schema in the form of a fixed point algorithm for the recognition of any variable definitions that always reach in or out of a basic block in a program.
For a node \(n \in V_{\text{CFG}}\), representing a basic block in the CFG, \callable[n]{Defines} being the set of variables \(x_n\) defined in \(n\) and \callable[n]{Pre} the predecessor nodes, the following equations set the reaching definitions of node \(n\)

\begin{align*}
  \text{\textbf{gen}}(n) &:= \text{\callable[n]{Defines}} \\
  \text{\textbf{kill}}(n) &:= \bigcup_{v \in V_{\text{CFG}}} x_v, \forall x \in \text{\callable[n]{Defines}} \\
  \text{\textbf{ReachIN}}(n) &:= \bigcup_{p \in \text{\callable[n]{Pre}}} \text{\textbf{ReachOUT}}(p) \\
  \text{\textbf{ReachOUT}}(n) &:= \text{\textbf{gen}}(n) \cup (\text{\textbf{ReachIN}}(n) \setminus \text{\textbf{kill}}(n))\\
\end{align*}

Having the two necessary representations of both a CDG and a DDG, a formal outline of a PDG is the tuple \(\langle V_{\text{PDG}}, E_{\text{PDG}} \rangle\), with \(\langle V_{\text{PDG}} = \langle V_{\text{CFG}}\) and \(E_{\text{PDG}} = E_{\text{CDG}} \cup E_{\text{DDG}}\).

There are many types of program slicing~\cite{DBLP:journals/csur/Silva12}, but the most significant ones are backward slicing and forward slicing.
As any other kind, these techniques are done using one statement as slicing criterion and may represent the statement that influence or are influenced inside the sliced program, respectively.
To do them, in both cases one must start a graph traversal from the slicing criterion, and look for either predecessor nodes (in backward slice) or successor nodes (in forward slice).
A program slicing is usually bounded by the entry and exit nodes of a PDG, but custom stopping conditions can always be applied.
The final slicing is obtained by removing any node that was not visited throughout the program slicing.

% TODO - add example figure of forward and backward slicing

\newpage
\section{Pynguin}

The \textbf{PY}tho\textbf{N} \textbf{G}eneral \textbf{U}n\textbf{I}t test ge\textbf{N}erator (Pynguin) is a Command Line Interface software intended for the automatic generation of unit tests for Python modules.
Its way of generating these Unit tests, is by using a different selection of search-based algorithms oriented to the test generation task, taken straightforward from the literature and implemented according to the internal representation in an Object-Oriented paradigm.

As mentioned previously, the main core of Pynguin's test representation is the abstract class \classname{Statement}, and its subsequent wrapper classes \classname{TestCase}, containing a list of \classname{Statement} instances, and \classname{TestCaseChromosome}, containing a \classname{TestCase}.
In the prior one, the next child classes available are \classname{AssignmentStatement}, a direct assignment of any kind of reference to a variable (e.g.~foo.bar = var\_1); and \classname{VariableCreatingStatement}, representing anything that can be assigned to a variable such as primitive types, collection types, fields, or any type of call (e.g.~int\_1~=~1, list\_1~=~\callable{list}, foo~=~Foo\((\text{int\_1,~list\_1})\)).
The following hierarchy of these two can be represented by Figure~\ref{fig:hierarchy}.

\begin{figure}[htb]\label{fig:hierarchy}
  \centering
  \includegraphics[width=1\textwidth]{Figures/statement_hierarchy2.png}
  \vspace*{0.5cm}
  \caption{Hierarchy of the \classname{Statement} class.}
\end{figure}

One class that assures Pynguin's easiness at the time of expanding it, is the abstract class \classname{GenerationAlgorithm}, which works as a layout for the insertion of any new heuristic of algorithm into the tool's current options, for the generation of \classname{TestSuiteChromosome}, another class wrapper for a list of \classname{TestCaseChromosome}.

In the context of evolutionary and genetic algorithms, the \classname{Statement} class has an abstract method called \callable{mutate}, which depending on the instance's final class, mutates either the primitive value, or the references that are encapsulated in it.

About the available algorithms\footnote{The work by Campos et al.~\cite{DBLP:journals/infsof/CamposGAFEA18} offers a further review and analysis of the different algorithms available in Pynguin.}, Pynguin has a total of 7: Random, Random Test Case Search, Random Test Suite Search, Whole Suite, MIO, MOSA and DynaMOSA.\@
From these, the experimental phase considered the last 4, because those implement either an archive or a dynamic chromosome population, which is necessary for the extension.

\begin{figure}[tb]
  \centering
  \includegraphics[width=.99\textwidth]{Figures/pynguin.png}
  \caption{Pynguin structure~\cite{DBLP:conf/icse/LukasczykF22}}\label{fig:pyn}
\end{figure}

Pynguin's workflow structure is presented in Figure~\ref{fig:pyn} and consists of a sequence of processes, starting with a Python Module as input, and getting a Python Test Module as output.
A general Pynguin execution begins by statically analysing the MUT in order to fill a ``Test Cluster'' with all relevant information for the posterior test generation, such as regarding types and methods.
Then, Pynguin individually picks functions or methods from the Test Cluster in order to generate a population (depending on the algorithm) of the \classname{TestCaseChromosome} class, firstly by generating a certain amount of either primitive, collection or instance statement used as parameter arguments, and then by generating the respective callable, owner of the different branches or lines to cover.
Once the final test suite is set, Pynguin tries to generate assertions for its test cases while creating mutations for themselves with MuyPY, to then export the final tests into new modules.

Pynguin also offers post generation processes, such as the deletion of redundant references (non-used variables) at the Chromosome level, or the generation of coverages reports for the output test files and spreadsheets with relevant information about the test generation, in HTML and CSV format respectively.

\newpage
\subsection{MOSA}

The Many-Objective Sorting Algorithm~\cite{DBLP:conf/icst/PanichellaKT15} (MOSA), is an improvement to the classic Multi-objective Optimization algorithms (MOA) such as the Non-dominated Sorting Genetic Algorithm II (NSGA-II) or the Strength Pareto Evolutionary Algorithm (SPEA2), and applied taking into account some specific draw backs of the branch coverage problem in test generation, such as the great computational effort needed by these classic algorithms when working with a number of objectives bigger than an order of magnitude of 10.
Another challenge that MOSA can overcome, is the selection and persistence over a non-dominated sort of those tests that have the best branch distance regarding non-covered goals.

Algorithm~\ref{alg:MOSApseudo} shows an overview of MOSA's pseudocode, which start with the generation of an initial random population.
Then, in every iteration an offspring is computed by means of arbitrary selection, crossover and mutation methods, which later is sorted into Pareto frontiers using Preference Sorting.
This procedure sets into the first frontier the best tests according to a preference criterion (usually the desired coverage), and then realizes a common Fast Non-dominated Sort over the rest of tests.
Finally, MOSA finishes every iteration of itself filling the next population with the right amount of tests from all the available Pareto frontiers.
About the MOSA implementation in Pynguin, it follows the same structure as the one in Algorithm~\ref{alg:MOSApseudo}.

\algrenewcommand\algorithmicrequire{\textbf{Input}}
\algrenewcommand\algorithmicensure{\textbf{Output}}


\begin{algorithm}[h!]
  \centering
  \caption{MOSA Pseudocode}\label{alg:MOSApseudo}
  \begin{algorithmic}[1]
    \Require~Stopping condition \(C\), Set of program targets \(B\), Population size \(M\)
    \Ensure~A test suite \(T\)
    \State~\(A, P_t \gets \{\}, \text{GenerateRandomPopulation}()\) 
    \While{\(\neg C\)}
      \State~\(Q_t \gets \text{GenerateOffspring}(P_t)\)
      \State~\(\text{UpdateArchive}(A, P_{t+1})\)
      \State~\(R_t \gets P_t \cup Q_t\)
      \State~\(F_x \gets \text{PreferenceSorting}(R_t)\)
      \State~\(P_{t+1} \gets \emptyset \)
      \State~\(d \gets 0\)
      \While{\(|P_{t+1}| + |F_d| \leq M\)}
        \State~\(\text{CrowdingDistanceAssignment}(F_d)\)
        \State~\(P_{t+1} \gets P_{t+1} \cup F_d\)
        \State~\(d \gets d + 1\)
      \EndWhile\@
      \State~\(\text{CrowdingDistanceSort}(F_d)\)
      \State~\(P_{t+1} \gets P_{t+1} \cup F_d[1\: (M - |P_{t+1}|)]\)
    \EndWhile\@
  \end{algorithmic}
\end{algorithm}

\newpage

\subsection{DynaMOSA}\label{sec:dynamosa}

DynaMOSA comes as a further extension of MOSA by Panichella et al.~\cite{DBLP:journals/tse/PanichellaKT18}, with the addition of a dynamic selection of the targets to be included into the first Pareto Frontier when executing the preference sorting.
The idea of including a dynamic selection of targets starts from the representation of a program into a CFG.
With it, is possible to note how those uncovered targets that have one or more already reached target predecessor are redundant to try to cover, because they always have objective score of \(f_r = \text{BranchDistance}(b) + 1\), where (\(+1\)) represents the approach level or the distance in the CFG to reach the successive target.
This means that every target can be prioritized into a hierarchy according to the definition of a CDG.

The pseudocode of DynaMOSA is shown in Algorithm~\ref{alg:DynaMOSApseudo}, with the only differences to MOSA being found in the lines 1, 5 and 9 when calling the set of an initial set of non-dependant targets and the dynamic update of these.
The \(\text{UpdateTarget}()\) procedure, is applied inside Pynguin by implementing a class for, a Control Flow Graph (named \classname{CFG}), as a wrapper of the \classname{ControlFlowGraph} class from the Python \classname{bytecode} module; and a Control Dependence Graph (called \classname{ControlDependenceGraph}).
Both of these classes are a fundamental part for the proposed heuristic implementation.

\begin{algorithm}[h!]
  \centering
  \caption{DynaMOSA Pseudocode}\label{alg:DynaMOSApseudo}
  \begin{algorithmic}[1]
    \Require~Stopping condition \(C\), Set of program targets \(B\), Population size \(M\), Control dependence graph \(G\), \(\phi\)~Map between edges of \(G\) and targets
    \Ensure~A test suite \(T\)
    \State~\(U^* \gets \text{NonDependentTargets}(B)\)
    \State~\(A, P_t \gets \{\}, \text{GenerateRandomPopulation}()\)
    \State~\(\text{UpdateTargets}(U^*, G, \phi)\)
    \While{\(\neg C\)}
      \State~\(Q_t \gets \text{GenerateOffspring}(P_t)\)
      \State~\(\text{UpdateArchive}(A, P_{t+1})\)
      \State~\(\text{UpdateTargets}(U^*, G, \phi)\)
      \State~\(R_t \gets P_t \cup Q_t\)
      \State~\(F_x \gets \text{PreferenceSorting}(R_t)\)
      \State~\(P_{t+1}, d \gets \emptyset, 0\)
      \While{\(|P_{t+1}| + |F_d| \leq M\)}
        \State~\(\text{CrowdingDistanceAssignment}(F_d)\)
        \State~\(P_{t+1} \gets P_{t+1} \cup F_d; d++\)
      \EndWhile\@
      \State~\(\text{CrowdingDistanceSort}(F_d)\)
      \State~\(P_{t+1} \gets P_{t+1} \cup F_d[1\: (M - |P_{t+1}|)]\)
    \EndWhile\@
  \end{algorithmic}
\end{algorithm}

\newpage

\subsection{Whole Test Suite}

The Whole Test Suite generation algorithm is an instance of a genetic algorithm that, as a result of custom operators, tries to optimize the computational effort of target coverage by evolving a population of test suites in a higher level of abstraction, compared to the other reviewed algorithms so far.
One of the main problems at the time of approaching test generation as a set of individual goals, is the case of encountering unreachable guards due to their recognition being an undecidable problem, and therefore it is not possible for the algorithm at hand to filter these elements in a trivial way.

Besides the preferred code coverage as fitness function, which is evaluated by running all test cases in a test suite and adding up all covered targets and distances, a secondary fitness function used is the size of the test suite, specifically for the forwarding of offspring chromosomes to further generations.
Algorithm~\ref{alg:WTSpseudo} shows Whole Test Suite's pseudocode, whose first relevant place is line 7 holding the crossover operator, that chooses a random value \(\alpha \in [0.0, 1.0]\) and generates 2 test suite offspring from the selected parents by splitting them both at position \(\alpha \times \text{Length}(p)\) and interchanging the segments.
Line 11 locates the mutation operator, where the algorithm mutates both test suites and test cases with a probability inverse to their own size, meaning that statistically only one of each are mutated in every iteration.
For test suites, with probability \(\sigma\) a new test case is added iteratively until the mutation fails.
Concurrently for test cases, three mutations can be applied with probability \(\frac{1}{3}\) each; remove, change or insert a statement.
The final novelty from this algorithm between lines 12 and 27, is the mechanism of generational forwarding already mentioned, which basically intends to bring the offspring to the next iteration's population only if either it has better fitness value, or if the fitness is equal to the one from the parents and the size of the offspring is much smaller.

In terms of the Whole Test Suite generation algorithm's implementation in Pynguin, it was done by setting a field in the \classname{WholeSuiteAlgorithm} class called \field{\_population}, containing a list of \classname{TestSuiteChromosome} objects.

\newpage

\begin{algorithm}[h!]
  \centering
  \caption{Whole Test Suite Pseudocode}\label{alg:WTSpseudo}
  \begin{algorithmic}[1]
    \Require~Stopping condition \(C\), Crossover probability \(P_c\)
    \Ensure~A test suite \(T\)
    \State~\(S \gets \text{GenerateRandomPopulation}()\)
    \While{\(\neg C\)}
      \State~\(Z \gets \text{Elitism}(S)\)
      \While{\(|Z| \neq |S|\)}
        \State~\(p_1,\, p_2 \gets \text{ParentRankSelection}(S)\)
        \If{\(P_c > \text{RandomFloat}()\)}
          \State~\(o_1,\, o_2 \gets \text{Crossover}(p_1,\, p_2)\)
        \Else\@
          \State~\(o_1,\, o_2 \gets p_1,\, p_2\)
        \EndIf\@
        \State~\(\text{Mutate}(o_1,\, o_2)\)
        \State~\(f_p \gets \text{MinimumFitness}(p_1,\, p_2)\)
        \State~\(f_o \gets \text{MinimumFitness}(o_1,\, o_2)\)
        \State~\(l_p \gets \text{Length}(p_1) + \text{Length}(p_2)\)
        \State~\(l_0 \gets \text{Length}(o_1) + \text{Length}(o_2)\)
        \State~\(T \gets \text{BestIndividual}(S)\)
        \If{\(f_o < f_p \lor (f_o = f_p \land l_o \leq l_p)\)}
          \For{\(o \in {o_1, o_2}\)}
            \If{\(\text{Length}(o) \leq 2\times \text{Length}(T)\)}
              \State~\(Z \gets Z \cup {o}\)
            \Else\@
              \State~\(Z \gets Z \cup {p_1 \text{or} p_2}\)
            \EndIf\@
          \EndFor\@
        \Else\@
          \State~\(Z \gets Z \cup {p_1, p_2}\)
        \EndIf\@
      \EndWhile\@
      \State~\(S \gets Z\)
    \EndWhile\@
  \end{algorithmic}
\end{algorithm}

\newpage

\subsection{MIO}

The Many Independent Objective (MIO) algorithm~\cite{DBLP:journals/infsof/Arcuri18} consists of a modification of the default (1 + 1) evolutionary algorithm, whose author came up with by taking the most notable characteristics of both MOSA and Whole Test Suite, in terms of exploration and exploitation of the search landscape.

One of these limitations, is the presence of a fixed size population that may not be big enough depending on the size of the System Under Test (SUT), the number of objectives to cover, or would not work optimally if too many redundant test have been introduced into it.
To fix this, MIO maintains an archive of test populations per target and sets a maximum amount of tests to be introduced to it.

MIO's core behaviour is represented by the existence of 3 main parameters; \(n\), being the number of tests per target to be stored in the archive; \(P_r\), the probability of introducing a randomly generated test into the population; and \(m\), the maximum amount of mutations to be applied.
To these, the algorithm also adds a fourth parameter \(F\), whose main purpose is to split the stopping conditions' budget in order to have two main phases with different sets of \(n\), \(P_r\) and \(m\).
Within the three parameters, it is stated that the first two are the more relatives ones at the time of controlling the exploration/exploitation.\@

A pseudocode of MIO is presented in Algorithm~\ref{alg:MIOpseudo}, which compared to Pynguin's implementation, only differs from the original algorithm's idea in the integrations of both \(T\) and \(A\) into a single data structure.

\begin{algorithm}[h!]
  \centering
  \caption{MIO Pseudocode}\label{alg:MIOpseudo}
  \begin{algorithmic}[1]
    \Require~Stopping condition \(C\), Population size limit \(n\), Start~of~focused search \(F\)
    \Ensure~Archive of optimized individuals A
    \State~\(T, A \gets \text{SetOfEmptyPopulations}(), \{\}\)
    \While{\(\neg C\)}
      \If{\(P_r > \text{RandomFloat}() \text{or IsEmpty}(T)\)}
        \State~\(p \gets \text{RandomIndividual}()\)
      \Else\@
        \State~\(p \gets \text{Mutate}(\text{SampleIndividual}(T))\)
      \EndIf\@
      \ForAll{\(k \in \text{ReachedTargets}(p)\)}
        \If{\(\text{IsTargetCovered}(k)\)}
          \State~\(\text{UpdateArchive}(A, p)\)
          \State~\(T \gets T \setminus \{T_k\}\)
        \Else\@
          \State~\(T_k \gets T_k \cup \{p\}\)
          \If{\(|T_k| > n\)}
          \State~\(\text{RemoveWorstTest}(T_k)\)
          \EndIf\@
        \EndIf\@
      \EndFor\@
      \State~\(\text{UpdateParameters}(F, P_r, n, m)\)
    \EndWhile\@
  \end{algorithmic}
  \end{algorithm} 

\newpage

\section{GBOS Heuristic}

The Graph Based Object Synthesis heuristic is the name given to the methodology presented by Yun et al.~\cite{DBLP:conf/sigsoft/0001O00D21} for the generation of complex object instances to be used as arguments in methods to be tested in a test case of the Java programming language.
One of the motivations for this research study was the possible non-continuity of test cases' search space with respect to coverage, when some related targets have the need to evaluate multiple conditions on an object instances before even getting an actual distance to them.
% TODO - maybe reference the example in the introduction?
The procedure done by this heuristic, consists of two main phases; the creation of an Object Construction Graph (OCG), that represents all relevant dependency flows of a method from its inputs to the actual target to be reached; and the translation of the OCG into an actual test case template, that serves as an initial search point for the test generation algorithm.

\subsection{OCG Generation}\label{subsec:ocg_gen}

For the generation of the OCG, one must start with the generation of the Program Dependency Graph (PDG), a complete representation of all control and data dependencies of a program.
As stated in Section~\ref{sec:slicing}, all dependencies of a program are represented with the aforementioned CDG and DDG.\@

The necessary PDG for the OCG's building process must also be interprocedural until a certain threshold level \(t_{\text{dep}}\), considering that one or many of the target method's input objects may be modified by a subsequent call.
This new requirement is met by just adding to the main PDG new edges from the correspondent callee's basic block to the entry node of the other method or function PDG, which also needs to be generated beforehand.

After having the full interprocedural PDG, a node containing a branch \(b\) (chosen target for the heuristic) must be used as entry criteria to apply a backward slice and either a node with an instruction reading a method input, a node with an instruction reading a global variable or an entry node, as stopping criteria.

The following step, is to analyse the sliced graph and link directly every instance of a node loading the complex method parameter objects with any node from the interprocedural methods' PDGs that retrieve a field or access an array, so the relevant variable state of these objects can be recorded flawlessly throughout the graph.

Finally, the sliced PDG is sliced further in a forward manner, using the object method inputs as slicing criteria and any field or array access as stopping condition, in order to leave only relevant paths for the heuristic's second phase.

\subsection{OCG Translation}

Once the OCG generation process is complete, its translation into actual code is done by including extra operations to a generic Breadth-first graph traversing algorithm.
The first step, is to initialize a test case \(t\) with a call to the target branch's method (and its respective object in case it is not static) together with an empty node-statement map.
Then, every time a relevant node is reached, such as the ones containing instructions of parameter loading, or attribute/array accessing, a corresponding object creating statement or attribute setting mechanism is added into the test and saved in the node-statement map.
The only case in which a field is called with a getter instead of a setter, is when the correspondent node is not a leaf, because this means that there might be a successor node either containing an array access or getting an object type field.
The purpose of the previously mentioned map, is to have direct reference of any object input at the time of setting its fields.

Considering the static typing of Java, one last matter to take in mind is the setting of non-public and non-static fields, which have to be managed through appropriate getter or setter methods.
This procedure is done by analysing all possible methods \(M = {m_1, m_2, \dots, m_k} \) in the correspondent class and building a directed call graph between them with paths \(P = \langle m_s, \dots, m_e\rangle\), where \(p_s\) represents the starting method, \(p_e\) the exit method that finally sets the field and the rest, the methods in between.
Using these two definitions, it is possible to approximate the likelihood of every method to set the field correctly with a metric composed of the number of parameters \(N_v\) and the complexity of the path \(P\).
For computing the path complexity, it is also necessary to check for every node in the path, what the ration between branches \(B'\) that access the next node in the path versus the total branches \(B\) is.
The path score is given by the equation
\begin{equation*}
  \text{score}(P) = (N_v + 1)\left(\frac{\sum_{s}^{e} \frac{|B'|}{|B|}}{|P|}\right)
\end{equation*}
and the sampling probability of a method \(m_i\) can be calculated by normalizing the maximum path score of every path that contains \(m_i\).

\chapter{GBOS Pynguin Extension}\label{chap:implementation}

Part of the preparations for the experimental phase and contributions of this thesis, is the creation of a functional extension of Pynguin inspired in the GBOS heuristic, that had to be developed using only the information available in its respective paper~\cite{DBLP:conf/sigsoft/0001O00D21} and some hints given by the authors of it.
The presented Python implementation followed a three-step scheme at Python Bytecode level, which aimed to be as loyal as possible to the original Java version while taking into account the differences between both languages.

\section{PDG Generation}

The development process started with the creation of \classname{DataDependenceGraph}, a new subclass of the already implemented \classname{ProgramGraph}, which serves as a superclass for both (also already available) \classname{CFG} and \classname{ControlFlowGraph}.
As its name hints, this new graph class encapsulates all necessary methods and further elements that help to collect all relevant data usages and definitions for the later generation of the reaching definitions sets.
Another crucial information obtained with the generation of the DDG through this class, are the methods and functions called along the program's control flow.  

In a first instance, one of the prerequisites for the coding of a DDG in Pynguin, was to split every node's basic block in the CFG class, into nodes with singular instruction.
If this had not been done, the final DDG would have set many variable uses and definitions in the same basic block node, simplifying the final graph to a trivial level that would have not worked in the further steps of the heuristic.
An example of this is shown in Figure~\ref{fig:ddg_example}.

\begin{figure}[htb]
  
\end{figure}

\begin{figure}[htb]
  \minipage{0.3\textwidth}
    \inputminted{python}{Figures/basic_block.py}
  \endminipage\hfill
  \minipage{0.65\textwidth}
    \includegraphics[width=\textwidth]{Figures/DDG.png}
  \endminipage\hfill
  \caption{A function \callable{foo} (left), a DDG of the function generated from a basic block CFG (middle) and another DDG, but generated from a single instruction CFG (right).}\label{fig:ddg_example}
\end{figure}


Then, the program's stack frame evolution is statically simulated and analysed to check which nodes have an instruction that defines or uses any variable names, or what functions or methods are called.
Opposite to off-the-shelf implementations of a stack frame analyser for Java (e.g. Java ASM\footnote{https://asm.ow2.io}), the used and defined data variables were collected every time they were either popped and served as operand for an instruction (e.g. BINARY\_ADD), and stored in the named stack or loaded back again in the stack frame (e.g. STORE\_FAST or LOAD\_FAST). 
This functionality was put into practice thanks to the classes \classname{DefUseAnalyzer} and \classname{Frame}, that traverse the CFG in a breadth-first manner and execute the corresponding single instruction respectively.

The way this works in the method \callable{DefUseAnalyzer.analyse}, is that every entry node starts with an empty \classname{Frame} with no variables in its stack.
Then, each non-entry node copies the frame from the first of its successors, execute the current instruction (with the method \callable{Frame.execute}) and set this modified copy as its own.
All the instructions' stack effects can be found in Python's developer documentation\footnote{https://docs.python.org/3.10/library/dis.html}.
As a design decision, the GBOS heuristic works with the bytecode instructions of Python version 3.10, because: it is one of Pynguin's supported versions; it restricts the amount Python Bytecode instructions needed to be implemented in the stack modifying match-case statement found in the \callable{Frame.execute} method; and it is stated in the GBOS paper~\cite{DBLP:conf/sigsoft/0001O00D21} that the approach is ``applicable in more general cases'', which implies the heuristic to work in any level of abstraction.

% TODO - Add an example of how frame works. method is called execute()

Once the \classname{DefUseAnalyzer} finishes generating two dictionaries (node, [variables]) for the uses and definitions, the DDG generation procedure copies every node in the given CFG, and proceeds to traverse once again the CFG in a breadth-first manner to apply the reaching definitions equations presented in Section~\ref{sec:slicing} to obtain the edges of the new graph.

With a program's DDG already generated, the CDG is obtained with the already implemented methods to do so, and the previously modified single-instruction CFG.\@
Both of these graphs are the base of the computation of an instance of the class \classname{ProgramDependenceGraph}, another subclass of \classname{ProgramGraph}, that initially just combines the edges of both CDG and DDG.\@
A second procedure realized as part of the PDG's computation, is the use of Pynguin test cluster and the (node, callable) dictionary obtained while getting the stack frame information, for the computation of subsequent called methods' PDGs.
This is done in a recursive manner and complying to a certain recursion level \(t_{\text{dep}}\), that works as a newly introduced Pynguin CLI argument. 

At this point, the interprocedural PDG is ready to be backward sliced with the necessary stopping conditions, contextualized to Python's bytecode instructions.
The slicing begins with the single instruction node in the PDG that contains the target branch, and it stops when either the node loads one of the target method's parameters (e.g.\@it is the first LOAD\_FAST that has one of the target method's parameters as argument) or it has a LOAD\_ATTR instruction; the node has a LOAD\_GLOBAL instruction; or the node has no predecessors (it is an entry node).

\section{PDG Interprocedural Analysis}

The interprocedural analysis is the next step to follow, and it starts by iterating over the list of know calls and selecting only the methods to check for the presence of an instruction chain 
\[\begin{pmatrix} \text{LOAD\_FAST} \\ \text{or} \\ \text{LOAD\_ATTR} \\ \text{or} \\ \text{BINARY\_SUBSCR} \end{pmatrix} \rightarrow \text{LOAD\_METHOD} \rightarrow \text{CALL\_METHOD}\]
that represents the load of the object instance, the load of its method and the call of it.
Within the scope of the current study, the analysis of function type callables (not belonging to any object or class definition) is left as a recommended future work subject, considering that working with a dynamically typed language is already challenging when applying a heuristic designed for a statically typed language (such as Java) and that there is not a reliable way of knowing how to link the arguments of a called function (probably loaded before a LOAD\_FUNCTION instruction) to their respective local variables inside the sub graph of the interprocedural PDG.\@

Then, from the first instruction node of the chain (that represents the callee of the method), it is needed to look into the sub-PDG of the called method for the first ``LOAD\_FAST self'', which represents the same object instance and might have a successor node LOAD\_ATTR that needs to be linked to the first node in the initial chain.

Figures~\ref{fig:inter_analysis} represents the outcome of an interprocedural analysis applied to the OCG of the method \callable{checkRules} from Listing~\ref{lst:1}.
There, the section marked as ``Interprocedural PDG'' corresponds to the sub-PDG of the method \callable{getActor} which is called at the node ``CALL\_METHOD 0''.
What the analysis does here, is replacing the method call edge into a direct data-dependency edge between ``LOAD\_FAST action'' (parameter input object) and ``LOAD\_ATTR \_actor'' (object field).

\begin{figure}[htb]
  \minipage{0.49\textwidth}
    \includegraphics[width=\linewidth]{Figures/before_analysis.png}
  \endminipage\hfill
  \minipage{0.49\textwidth}
    \includegraphics[width=\linewidth]{Figures/after_analysis.png}
  \endminipage\hfill
\caption{An example of OCG before and after the interprocedural analysis}\label{fig:inter_analysis}
\end{figure}

A forward slicing is then performed into the PDG, starting from the parameter loading nodes and stopping at relevant nodes (those with LOAD\_ATTR or BINARY\_SUBSRC instructions) fulfilling certain conditions.
The conditions for this operation are, that the relevant nodes stop the slicing, if and only if they do not have any more child nodes that are also relevant.

\section{OCG Translation}

When the OCG generation phase is ready, the last step of the GBOS heuristic is to translate it, in this case, into a \classname{TestCaseChromosome} instance.
The creation of it begins with a default empty chromosome, to which two possible main statements are added: a call statement of the target method or function, and positioned before that, a variable creating statement with the class containing that method.
This last statement is ignored if the target callable is a function.

Before starting to generate statements, the target method is analysed one more time, using Python's AST\footnote{https://docs.python.org/3.10/library/ast.html} and Astroid\footnote{https://pylint.pycqa.org/projects/astroid/en/latest/} libraries, in the look for any useful type annotation that might be inside the \classname{FunctionDef} and \classname{ClassDef} of the target method's module AST.\@ 
Then, similar to most of the previous graph traversing steps, the OCG is covered in a breadth-first manner while maintaining an (ID,~variable~reference) map to keep track of all the created object instances and set fields.
The keys of the map are set to ID = parentID~+~'.'~+~instruction.arg, with the idea  to avoid redundancy at the time of adding statements to the test case.
An example of this, is the possible appearance of a node with the instruction ``LOAD\_FAST state'' multiple times within the OCG, which could lead to generating multiple object instances of type \classname{GameState} even though it is necessary to generate it only once.
This implies, that only a variable reference is stored in the map for the key ``state'' at the first appearance of ``LOAD\_FAST state'', or for the key ``state.\_players'' at the first appearance of the chain ``LOAD\_FAST state \(\rightarrow\) LOAD\_ATTR \_player''.

Although the OCG was sliced thoroughly during its generation, most  of the nodes in it are still irrelevant for statement creation.
This, because at this point the only instructions that are interesting for the test case template are those that load object instances such as the ones from method's inputs, load attributes of the aforementioned object inputs, and access possible collection-like data types from Python, which might have further complex objects as elements.

For nodes that encapsulate a \textbf{LOAD\_FAST} instruction, an object instance is created in case the node's argument is one of the target method's parameters, and it is the first appearance of it.

With \textbf{LOAD\_ATTR} nodes, first it is checked if there is any existent key in the variable reference map regarding one of the current node's predecessors. 
If there is an already created object instance, an assignment statement is added into the test case, with a field access of the object as the left-hand side of the assignment, and a new primitive or object variable reference as the right-hand side.
Regarding the path score, used to select the adequate setter method in the heuristic's paper Java implementation, there was no need to apply it in this Python version, as Python does not have an applied concept of private fields.
At most, there is the developer community standard of using an underscore (``\_'') character before a field variable name to simulate it as private, but this does not have any real run-time implications.
One argument for implementing a non-arbitrary way to select a certain method to set fields, is that there is not a concrete way in the actual Python's literature, to know if a method modifies a certain field, other than the presence of the instructions LOAD\_ATTR, STORE\_ATTR or DELETE\_ATTR.\@
The usage of getters in the original Java version was replaced in this case with the variable reference map, which provides information regarding the position of the object instance inside the test case, the type information and the annotations, if available.
Overall, this is the main and only difference between this thesis' work and the original heuristic proposed by Lin et al.~\cite{DBLP:conf/sigsoft/0001O00D21}.

When encountering a \textbf{BINARY\_SUBSRC} instruction, the OCG is checked to look for a predecessor node with an already generated collection variable that contains object elements.
In both ``LOAD\_FAST'' and ``LOAD\_ATTR'', the list variables are initialized with a few references of the same variable as elements (1 to 10, as arbitrary values), and when these are object types, the used instance is the one stored into the map of references.
In this context, this node only takes any available reference of the parent nodes, but as it needs to save it every time to start the chain of references, the index of this node in the reference map is the string version of its own index.

Once the test case template generation process is finished, the test case is executed with the available \classname{TestExecutor} instance, and the results are stored into the same \classname{TestCaseChromosome}.
Listing~\ref{lst:6} presents one of the cases generated when using the GBOS extension, showing particular similarity with the hypothetical test template proposed in Listing~\ref{lst:4}.

\begin{figure}
  \inputminted[linenos]{python}{Figures/template-result.py}
  \caption{Mutated test template obtained through the execution of Pynguin with the GBOS extension for the test of the example module in Listing~\ref{lst:1}\label{lst:6}}
\end{figure}

For every extended algorithm, the following steps are done

\begin{itemize}
  \item MIO:\@ the archive is updated with the implemented method of the archive instance \callable[\text{test\_template}]{MIOAlgorithm.\_archive.update}.

  \item MOSA and DynaMOSA:\@ the test template is first checked for any execution exception and then introduced into the '\_population' field of the \classname{AbstractMOSAAlgorithm} class, which is a list of \classname{TestCaseChromosome} elements.
  \item Whole Test Suite: same as for MOSA and DynaMOSA, the test template is checked for exceptions and then is introduced in one of the \classname{TestSuiteChromosome} available in the '\_population' field of the algorithm instance, \classname{WholeSuiteAlgorithm}.
\end{itemize}

About the new parameters introduced to Pynguin for the GBOS heuristic, there is the binary argument ``gbos'', meaning the usage of the extension; the numerical argument ``no\_change\_iteration'', stating the amount of iteration in which there may be no change in a certain goal in order to consider it eligible to create a new test template for; the float argument ``p\_app'', describing the probability of creating a test template for an eligible target; and the numerical argument ``max\_recursion\_level'', defining the maximum amount of recursion calls that can be made while generating the OCG.\@

One remark that must be taking into account with respect to the implementation, is that the GBOS extension relies completely on the assumption that the any time a complex object input is needed for the reaching of a branch target, all required type information about the parameters and relevant object fields will be available. 

\chapter{Evaluation}~\label{chap:evaluation}

The previous work on the implemented GBOS extension has shown that in a general case, using this heuristic improved significantly the coverage results of automatically generated tests of Java classes.
In the same way, it is relevant to study the behaviour of Pynguin's GBOS implementation in terms of coverage, while analysing the programming style of the Python modules of the testing set.
Therefore, the following research questions are formulated:

\begin{resq}
  Can the Graph-Based Object Synthesis extension outperform Pynguin in its original paper's experimental setup similarly as EvoObj did with EvoSuite?
\end{resq}

For the answer of this Research Question, a union of two previous test sets of Pynguin's research papers was put to the test to benchmark the performances of both Pynguin in the latest developer version (0.35.0) that the repository had before cloning it and the extension to the tool, presented by this thesis.
To avoid redundancy in the results, from the total amount of modules, only those that are not considered trivial stayed for the evaluation (more details about the test set in Section~\ref{sec:experimental_setup}).
Then, from the branch coverage results, a Mann-Whitney U test and the Cohen D's effect size measure were calculated in order to compare them to the results obtained by Yun Lin et al. 

\begin{resq}
  Is there a correlation between the difference in coverage results and the amount of object-like callable inputs of MUTs while using Pynguin and the GBOS extension?
\end{resq}

In this scenario, the same module set from the previous research question was used.
From the test set, the percentage of object-like inputs in methods and functions was obtained by analysing the AST of the MUTs with the following formula
\begin{equation}\label{eq:olr}
\text{OLR} = \frac{1}{n}\sum_{m_i \in M} \frac{o_i}{p_i}
\end{equation}
where \(M\) is the set of callables in a MUT with at least one object-like parameter, \(p_i\) the amount of parameters in \(m_i\), \(o_i\) the amount of object-like parameters in \(m_i\), and \(n\) the amount of methods in \(M\) (from now on, the metric is called the object-like ratio or OLR).
An object-like parameter references any method argument that has type information, and it is not either one of Python's primitive types or built-in collections.
One exception to this rationale, is when one argument is identified as a collection type that has object-like types as elements which, although it could be considered a recursive rule, it was limited to a first class wrapper.
Here, the OLR serves as a way to imply the presence of complex object inputs and see if the metric presents a correlation with the development of branch coverage when using or not the extension.
For those modules that stay in the test set for this research question after a filtering process, the \(\hat{A}_{12}\) Vargha-Delaney effect size metric is going to be computed in order to calculate the Pearson's and Spearman's correlation, and answer the research question. 

\section{Experimental Setup}\label{sec:experimental_setup}

For the experimental executions of Pynguin with the new GBOS extension, the data set of Python projects initially used as a benchmark for Pynguin's type tracing update, joined with a subset of Pynguin's EMSE submission test modules~\cite{DBLP:journals/corr/abs-2111-05003}, was taken as a base for the answer of both research questions.
This initial testing set is formed by 1047 modules over 44 projects.

In a first instance, are going to be filtered all modules that reach full branch coverage using the MIO algorithm without the GBOS extension, before 100 iterations or 600 seconds of execution time.
The filtering process left at this point 588 modules over the same 44 projects.
For the answering of the second research question, a final filtering criteria was applied by comparing the  results of the runs to their Import Coverage, and checking the object-like ratio described in Equation~\ref{eq:olr}.
Here, only those modules that have a final coverage that is different to their Import Coverage value or have a OLR greater than \(0.5\) are kept.

Regarding the execution parameters of Pynguin, 8 main configurations are used to cover the 4 extended algorithms, and ether the usage or not usage of the GBOS extension itself.
The configurations that use the WHOLE\_SUITE argument, also needed to set the parameter ``--use\_archive'' to true, as the extension uses the archive of each \classname{GenerationAlgorithm} to obtain the branch goals.
From the newly introduced parameters, only ``--no\_change\_iterations'' was tuned in a hand-made manner while having the first test runs as reference.
In these (and similar to the results finally used for the analysis), the runs using the MIO algorithm reached mean algorithm iteration values close to an order of magnitude between \(10^5\) and \(10^6\), contrary to the rest of algorithms, which reached values closer to an order of magnitude of \(10^4\).
Because of this, the mentioned parameter was left at the default value of \(1000\) for MIO, and lowered to \(300\) for the other algorithms.
The maximum execution time was set to 600 seconds, and the remaining parameters were set to Pynguin's default values, as no further Parameter Tuning was applied.

As a way to make visible the possible execution time trade-off between the different phases of the GBOS extension and Pynguin's own search time output value, and apply the subsequent analysis individually, new output variables were introduced into the post-search process.
These are ``GBOSDecisionTime'' and ``GBOSGenerationTime'', representing the amount of time the GBOS extension needed for checking if the conditions to entering the generation phase were met, and for the test template generation phase itself, respectively.
In that sense, the time measure stored in the new variables is not counted as part of the already implemented ``SearchTime'', but are part of ``TotalTime''.

\section{Threats to Validity}

The threats to validity present in the study can be generalized into two main categories: threats to \textbf{internal validity}, to \textbf{external validity} and to \textbf{construct validity}.

The GBOS extension's \textbf{internal validity} is threatened by the uncertainty of how many possible cases of what is considered a complex callable input are covered, including the amount of recursions in wrapper classes that are taking into account while computing the type annotations, or the consideration of all wrapper classes available for Python in its 3.10 version.
The definition of the Object-like ratio also makes an internal threat, as it was used as an intermediary step into identifying the possible complex object inputs, stating that primitive types axiomatically are not objects nor complex, but at the same time it is not a determinant measurement.
An example of this is a variation of the method \callable[]{checkRules} from Figure~\ref{lst:1}, but taken from line 1 until before line 10, where the method would have an OLR of \(100\%\) but the unique target (branch of line 8) would not need any complex object in order to be reached.
The plausible presence of bugs in the multiple stages of the GBOS phase also threatens the internal validity, which is partially mitigated by the if-clauses that stop the whole process if one of the steps returns a None value (e.g.\ while constructing the PDG, OCG or the test template).
The necessity of both complex object inputs and full knowledge of the variables' type information in order to obtain any improvement in terms of branch coverage threatens the internal validity too.

With respect to the generalization of the benefits obtained from the presented extension, the \textbf{external validity} is threatened by the usage of Python modules that were already studied in previous studies of Pynguin, and might not represent the great majority of Python modules. 

About \textbf{construct validity}, the usage of the extension cannot be decided through Parameter Tuning, but only via the knowledge of each developer of the presence of complex object inputs in a method or function, which implies that no automatic tuning was made into the experimental phase, threatening the study's operationalization.

\iffalse

\section{General Results}

Figure~\ref{fig:iterations} reveals the distribution of algorithm iterations through the different configurations.
Similar to the previous graph, the results' comparison between Pynguin and the GBOS extension show almost no difference while observing the probability density, median position and interquartile position (except for the notorious presence of higher values in WHOLE\_SUITE).
The values for percentage difference between both approaches over the average value of plotted distributions, the different algorithms get \(-7.20\%\), \(-3.65\%\), \(-6.48\%\) and \(0.92\%\), in the same order from left to right as they are presented in the referenced Figure.

\begin{figure}[th]
  \centering
  \includegraphics[width=0.9\textwidth]{Figures/Results/AlgorithmIterations.jpg}
  \caption{Computed iterations between all 8 parameter configurations over the modules in the test set.}\label{fig:iterations}
\end{figure}

\fi

\section{Research Question 1}

\begin{figure}[b]
  \minipage{0.5\textwidth}
    \includegraphics[width=\linewidth]{Figures/Results/SearchTime.jpg}
  \endminipage\hfill
  \minipage{0.5\textwidth}
    \includegraphics[width=\linewidth]{Figures/Results/timeDist.jpg}
  \endminipage\@
  \caption{Search time distribution between the 8 parameter configurations and average values for new output variables of the GBOS extension.}\label{fig:times}
\end{figure}

Figure~\ref{fig:times} describes through two plots the overall search time and the time needed by the extension per phase.
The violin plot on the left, shows the Search Time distribution of the runs, revealing that both Pynguin and the extension use the full budget of 600 seconds for any computation that is not related to the GBOS extension.
The percentage change (e.g.~\(\frac{\text{GBOSValue} - \text{PynguinValue}}{\text{PynguinValue}}\times 100\)) for the average values of the distributions presented in this plot, are \(9.35\), \(6.92\), \(7.52\) and \(1.74\) respectively, from left to right.

Regarding the times that are considered while doing any GBOS-related operation, the bar plot on the right shows a comparison of the average values per module, per algorithm, of the time spent in deciding if the current iteration generates a test template or not (Decision Time), time spent generating the test templates (Generation Time), and the Search Time as baseline.
By looking at the sizes of the different sections of each algorithms' bars, it is safe to state that compared to the other options, MIO is the algorithm that is affected the most with the extra computational effort that the GBOS extension generates.
This can be explained by the lightweight nature of the evolution steps implemented in Pynguin, which are reduced to only selecting a new test case if necessary and mutating it the maximum amount of times possible depending on the algorithm's parameters.
Considering a comparison between the Search Time and the Decision Time, MIO obtains a percentage change of \(279.86\), while the other algorithms have values ranging between \(-87.48\) and \(-83.68\).
With respect to the Generation Time, the comparison using percentage change of MIO gives a value of \(346.54\), while the other algorithms have values ranging between \(53.69\) and \(74.75\).

The data reported in Table~\ref{tab:performance} shows the mean coverage of all tested modules per approach, the p-Values obtained through statistical Mann-Whitney U tests and Cohen's D effect size measures, in any pair of algorithm and runtime budget.
Generally, all p-Values are greater than \(0.05\) with respect to the sets of obtained branch coverage values in each module, accepting the null hypothesis that assumes no effect or statistical difference between the distributions.
Complementary, Figure~\ref{fig:coverage} shows the branch coverage results and their distribution through the runs of the 8 different parameter configurations.
The presented violin plots imply that in general, both approaches to the test generation for the set of Python modules got similar results, based on the form of the plots' probability density, the position of the median and the position of the interquartile range.
%For more precise data, when calculating the percentage difference between Pynguin and the GBOS extension over the average value of the plotted distributions; DYNAMOSA, MOSA, WHOLE\_SUITE and MIO get values of \(-0.27\%\), \(-0.14\%\), \(-0.28\%\) and \(-0.68\%\) respectively. 

\begin{figure}[tbh]
  \centering
  \includegraphics[width=0.9\textwidth]{Figures/Results/Coverage.jpg}
  \caption{Mean coverage between all 8 parameter configurations over the modules in the test set.}\label{fig:coverage}
\end{figure}

Nonetheless, it is noticeable that all values of Cohen's D effect size in Table~\ref{tab:performance} are negative, meaning that the best distributions of coverage values are found while using Pynguin without the GBOS extension.
This strongly implies that in a general Python developing scenario the usage of the GBOS extension is not advised nor makes any considerable improvement in coverage results, as the mean effect size found in the last time step budget of the runs gets a value of \(-0.32\) over all algorithms, contrary to the value of \(0.21\) (in favour of EvoObj) obtained by Yun Lin et al.\ over all the algorithms used in their experimental phase for the maximum budget of 300 seconds. 

\begin{table}[h!]
  \centering
  \begin{tabular}{lcccc}\toprule 
\multirow{2}{*}{Coverage Performance} & \multicolumn{4}{c}{Budget - 300[s] } \\ \cmidrule(lr){2-5}  
                                      & DYNAMOSA&MOSA&WHOLESUITE&MIO                         \\ \midrule 
GBOS                                  & \(0.62\)&\(0.62\)&\(0.62\)&\(0.61\)                       \\ 
Pynguin                               & \(0.62\)&\(0.62\)&\(0.62\)&\(0.62\)                       \\ 
p Value                               & \(1.0\)&\(1.0\)&\(0.98\)&\(0.93\)                     \\ 
Cohen's D Effect Size                 & \(0.01\)&\(0.01\)&\(-0.01\)&\(-0.2\)                       \\ 
\bottomrule 
\end{tabular}
  \centering
  \begin{tabular}{lcccc}\toprule 
\multirow{2}{*}{Coverage Performance} & \multicolumn{4}{c}{Budget - 450[s] } \\ \cmidrule(lr){2-5}  
                                      & DYNAMOSA&MOSA&WHOLESUITE&MIO                         \\ \midrule 
GBOS                                  & \(0.62\)&\(0.62\)&\(0.62\)&\(0.61\)                       \\ 
Pynguin                               & \(0.62\)&\(0.62\)&\(0.62\)&\(0.62\)                       \\ 
p Value                               & \(1.0\)&\(1.0\)&\(0.99\)&\(0.93\)                     \\ 
Cohen's D Effect Size                 & \(0.01\)&\(0.0\)&\(-0.01\)&\(-0.19\)                       \\ 
\bottomrule 
\end{tabular}
  \centering
  \begin{tabular}{lcccc}\toprule 
\multirow{2}{*}{Coverage Performance} & \multicolumn{4}{c}{Budget - 600[s] } \\ \cmidrule(lr){2-5}  
                                      & DYNAMOSA&MOSA&WHOLESUITE&MIO                         \\ \midrule 
GBOS                                  & \(0.6\)&\(0.6\)&\(0.57\)&\(0.59\)                       \\ 
Pynguin                               & \(0.61\)&\(0.61\)&\(0.58\)&\(0.62\)                       \\ 
p Value                               & \(0.74\)&\(0.75\)&\(0.7\)&\(0.76\)                     \\ 
Cohen's D Effect Size                 & \(-0.26\)&\(-0.22\)&\(-0.27\)&\(-0.52\)                       \\ 
\bottomrule 
\end{tabular}
  \caption{Coverage Performance of Pynguin and GBOS with different time budgets.}\label{tab:performance}
\end{table}

\begin{summary}{RQ1: Performance of GBOS over Pynguin}
  The presented GBOS extension for Pynguin \textbf{does not} outperform Pynguin itself while being evaluated over previously studied modules, similarly as EvoObj outperformed EvoSuite. 
\end{summary}

\newpage

\subsection*{Discussion}

With the results in Table~\ref{tab:performance}, it is shown once more how the extra computational effort needed while using the GBOS extension does not influence the obtained coverage results, at least as a rule of thumb.
Another possible factor that explains the lack of improvement in the coverage results, is the absence of a need for complex object inputs by the modules under test in order to reach all target branches.
This latter argument can be backed up partially with the results in Figure~\ref{fig:times}, which tell that even after the applied filtering, some modules still could reach a full branch coverage with the tests provided by an initial population of test chromosomes (e.g.~getting a Search Time of 0 [sec]).
In addition, the data per time budget step mostly says that during the coverage evolution of Pynguin, there is persistent difference in performance while using the extension, which can be attributed to the random nature of the Search Based algorithms, or to the inter-iteration time difference in the Search Time variable when using the extension, in the case of the final time step of 600 seconds.

Figures~\ref{fig:worst} and~\ref{fig:best} show the branch coverage evolution along the 600 seconds budget for those modules that obtained the best and worst effect size values on each algorithm.
In the first plot, during the whole search time budget it can be remarked how the GBOS extension has poorly evolves compared to Pynguin by itself, following the behaviour already found in Table~\ref{tab:performance}.
Contrarily, rather than showing a better performance, the second graph implies that when the best effect sizes where found, both approaches had an almost equal evolution in branch coverage over time, except for a few segments, as it can be seen in the lines for DynaMOSA, MOSA and Whole Test Suite.

In terms of number of runs successfully executed, the nature behind the obtained effect size values could have been influenced by the difference in available runs, which ranged between \(2\) and \(28\).
%The lower number of runs was mostly obtained when running Pynguin with the extension, existing cases in which this only obtained as few as a single possible run.
\begin{figure}[pbth]
  \centering
  \includegraphics[width=0.9\textwidth]{Figures/Results/worstES.jpg}
  \caption{Branch coverage evolution for the modules that obtained the worst Cohen's D effect size value in each algorithm.}\label{fig:worst}
\end{figure}

\begin{figure}[ptbh]
  \centering
  \includegraphics[width=0.9\textwidth]{Figures/Results/bestES.jpg}
  \caption{Branch coverage evolution for the modules that obtained the best Cohen's D effect size value in each algorithm.}\label{fig:best}
\end{figure}

\newpage


\section{Research Question 2}

Figure~\ref{fig:scatter} shows a scatter plot of the different modules that presented an OLR measure greater than \(50\%\) and their Vargha-Delaney \(\hat{A}_{12}\) effect size with respect with the two distributions available for each approach and algorithm.

\begin{figure}[bth]
  \centering
  \includegraphics[width=0.9\textwidth]{Figures/Results/scatterplot.jpg}
  \caption{Scatter plot of the \(\hat{A}_{12}\) effect size values versus the OLR value, for every module that reported a OLR value greater than \(0.5\)}\label{fig:scatter}
\end{figure}

To answer this research question, two types of correlations metrics were calculated; Pearson's correlation coefficient, for linear dependency; and Spearman's rank correlation coefficient, for monotonic relationship between the variables.
Both of these metrics are categorized depending on the magnitude and sign of their values, ranging from \(-1\) to \(1\) and stating a strong correlation for magnitudes closer to \(1\), a direct relation for positive values and an inverse relation for negative values.

The values obtained for each metric follow a repetitive behaviour for each algorithm:
\begin{itemize}
  \item the Pearson's correlation coefficient values ranged between \(0.09\) and \(0.19\), meaning no linear correlation between the variables OLR and \(\hat{A}_{12}\) effect size
  \item the Spearman's rank correlation coefficient values ranged between \(0.01\) and \(0.19\), meaning no monotonic correlation between the variables OLR and \(\hat{A}_{12}\) effect size
\end{itemize}

implying, that there is no correlation between the aforementioned variables, when using the GBOS extension in a general Python module scenario.

\begin{summary}{RQ2: Correlation of coverage improvement and OLR}
  The usage of the presented GBOS extension \textbf{does not} induce a correlated improvement in branch coverage results, when applied to Python modules that have a higher OLR metric.
\end{summary}

\subsection*{Discussion}

\begin{figure}[ptbh]
  \centering
  \includegraphics[width=0.9\textwidth]{Figures/Results/worstA12.jpg}
  \caption{Branch coverage evolution for the modules that obtained the worst Vargha-Delaney \(\hat{A}_{12}\) effect size value in each algorithm.}\label{fig:worsta12}
\end{figure}

\begin{figure}[ptbh]
  \centering
  \includegraphics[width=0.9\textwidth]{Figures/Results/bestA12.jpg}
  \caption{Branch coverage evolution for the modules that obtained the best Vargha-Delaney \(\hat{A}_{12}\) effect size value in each algorithm.}\label{fig:besta12}
\end{figure}

The obtained answer for the second research question develops the same ideas as the discussion of the first research question, stating that the majority of Python modules in the testing set either, do not include type information or the necessity for complex object inputs.
Following the same procedure as before, Figures~\ref{fig:worsta12}~and~\ref{fig:besta12} show the evolution in coverage results over the time budget, for those modules that obtained the worst \(\hat{A}_{12}\) effect size measure in the second research question.
Once more, and supported by all the data gathered so far, the results in coverage for this effect size measure prove that both approaches of Pynguin shows in each chosen algorithm, either no statistical difference or a gap so big in the amount of runs between distributions, that the results in effect size bring no conclusive information.
Examples of this are the plots for DYNAMOSA and MOSA in Figure~\ref{fig:worsta12}, which are instances were the \(\hat{A}_{12}\) is lower than \(0.5\) (meaning better results when using Pynguin without the extension) but the mean final coverage show two scenarios, one where the GBOS extension has a better final result and one where it does not.
Finally, this could have been influenced once more by the fact that in most of the extreme cases of effect size results (best and worst), Pynguin without GBOS could finish most of its 30 proposed iterations and with the extension it only could finish one or two. 

\newpage

\section{Further Analysis}\label{sec:further}

To give a brief description of the different modules that made up the test set, Figure~\ref{fig:olr-hist} shows a ranged histogram (with bins of length \(0.1\)) of Object-like ratios over the 588 executed modules with the continuous line representing the kernel density estimate.
From these, \(91.33\%\) of the modules, or 537 out of 588 have a ratio below \(50\%\).
This values might explain the null improvement when using the GBOS extension, as it was stated before, one of the prerequisites to make the implemented heuristic work, is to have full type information regarding the possible complex object inputs needed to reach the target branches.
Analogously, the OLR distribution does not mean that for the great majority of modules there are not any needed complex object input, but that only for those that have an OLR greater than \(0.0\%\) (and by chance include a target reachable by generating a test with a complex object input) the presented GBOS may be of help in the improvement of coverage results.

\begin{figure}[bth]
  \centering
  \includegraphics[width=0.9\textwidth]{Figures/Results/olr-hist.jpg}
  \caption{Histogram of Object-like ratio over the modules in the test set.}\label{fig:olr-hist}
\end{figure}

For a deeper understanding of the modules that reached the endmost effect size results in Figures~\ref{fig:worst},~\ref{fig:best},~\ref{fig:worsta12}~and~\ref{fig:besta12}, a final new developer output variable was added into the code of the GBOS extension, called ``GBOSFlags''.
This variable is a collection of three boolean values, which start with False as default, and change to True if any of the following conditions are met at any point during the test suite generation,
\begin{itemize}
  \item a PDG, or a non-sliced OCG is created without errors
  \item a test template is generated without exceptions raised
  \item a test template is added into the algorithm's population.
\end{itemize}
With this information, it was possible to understand that in the great majority of aforementioned modules, none of these three values was set to True, fact that further explains the coverage results and base the answers given to the proposed research questions.

The only different case of value for the output variable, was the module ``pkg\_resources.\_vendor.packaging.tags'', whose test template generation process managed to create a valid PDG, but not a valid test template.
Considering this case, questions arise about the element of the GBOS extension that made the second step fail, which is most likely due to a lack of information in terms of Data Dependency within the built inter procedural PDG.\@
One example of this information loss can be found in the presented implementation of the Frame execution when generating the DDG, specifically when the MUT has ``try-except'' or ``with'' clauses inside any of its basic blocks.
The problem with these logic rules comes, because their implementation is given (on the 3.10 Python version) with Python instruction that conditionalize the stack frame state on the presence of an Exception, e.g.~\textbf{SETUP\_WITH} and \textbf{SETUP\_FINALLY}.
An example of this behaviour is the snippet

\begin{minted}{python}
try:
    # A SETUP_FINALLY is emitted here, pointing to the except block
    pass
except:
    pass
\end{minted}

where the stack frame suffers no relative change in case of no Exception, or is restored to its state before the ``try'' block and pushes different Exception-related variables onto the stack (e.g.~the raised Exception, the address of the exception handler, etc.) before continuing in the ``except'' block on the contrary case.  
The procedure in this type of instruction for the development stage of the extension was to set the node's stack frame effect to be the one given by the successful execution.

With this case present in the OCG generation, it is non-trivial to obtain the use and definition places, and consequently it is not possible to fully define the reaching definition equations.
Sometimes, this problem is represented even by a runtime error, when an instruction of the extended OCG that is being executed, tries to do a POP operation on an empty stack frame.  
Nonetheless, in the experiment runs this error did not raise any Exception, which still allowed the heuristic to generate a non-finished OCG.\@ 

Regarding once more the example module presented in Figure~\ref{lst:1}, the line plots presented in Figure~\ref{fig:example_cov} show the evolution in branch coverage results over a Search Budget of 600 seconds for four cases of two characteristics each, either using or not the GBOS extension, and either using MIO and MOSA as examples of evolutionary and genetic algorithms.
In them, it is remarkable how none of the baseline Pynguin approaches could escape the search landscape local optima, which is attributed to the random nature prevalent in both algorithms, together with the limited search time.
Contrastingly, both GBOS approaches reached the full coverage before spending the whole search time budget, when using ``no\_change\_iteration'' arguments of 100 and 500 for MOSA and MIO respectively.
These latter experimental instances show a better performance in terms of sooner convergence when using the MIO algorithm, which is explained by the fact that although MIO was configured to generate test templates less often compared to MOSA, previous results showed that the amount of iterations that MIO performs is exponentially greater to other algorithms.
Moreover, the way MIO is implemented into Pynguin makes the execution of it prioritize the mutation of each iteration's single solution, thing that is strictly necessary in order to reach branch targets after generating a test code template. 

\begin{figure}[htb]
  \centering
  \includegraphics[width=0.9\textwidth]{Figures/Results/exampleCov.jpg}
  \caption{Coverage evolution while running Pynguin for the example module introduced in Chapter~\ref{chap:introduction}.}\label{fig:example_cov}
\end{figure}

\chapter{Related Work}\label{chap:related_work}

% TODO - change structure by approach (check Gordon's email)

Before commenting on the different tools, techniques and methods that have a place in the current state of the art of automatic test generation for programming languages, it is worth describing the types of testing and their objectives. The following types were investigated by one or more of the tools mentioned in this section. 

\begin{enumerate}
  \item Unit testing: a type of functional testing that tries to test individual or atomic units of a software project.
  \item Regression testing: a type of system testing, which tells if new functionalities of a system produce defects into the latest build of it.
  \item Fuzz testing: a kind of security testing that consists of providing random, unexpected or large amounts of data to a system, to review its way of behaving to it.
\end{enumerate}

The various approaches that are part of the current state-of-the-art for Automatic Test Generation can be categorized in two main types depending on what are they based on: Large Language Models (LLMs) and Search-Based Algorithms or Static Analysis (SB).

\section{LLM-Based}

\subsection{CodaMOSA}

One of the many tools presented in papers that have cited Pynguin in their research is CodaMOSA~\cite{DBLP:conf/icse/LemieuxILS23}, software created by Microsoft developers, has Pynguin as a base and extends its functionality calling OpenAI's Codex\footnote{https://openai.com/blog/openai-codex} API to get template suggestions for test cases that are in a coverage stall, similar to what this thesis works over.
For the development, the authors used only MOSA as the search algorithm because is the only one that allowed the usage of both line and branch coverage as fitness functions, on the version of Pynguin at the time (0.19.0).

The procedure done by CodaMOSA, queries prompts to Codex after the coverage has not changed in a certain amount of iterations, and from a set of target generations of chromosomes that it statistically chooses to generate a test from either a function, method or constructor.
After that, it deserializes the returned test case from Codex into the internal representation of Pynguin, while applying different heuristics, such as removing nested expressions and use single assignment.

For the benchmark, modules from 35 projects from the Pynguin paper were filtered so just those that do not fail to produce result and those that do not reach \(100\%\) coverage in less than a minute are experimented over.
The baselines for the experimental phase were Pynguin with MOSA and CodexOnly, which while being compared to CodaMOSA were statistically compared using the Mann-Whitney U-Test.
The results showed that CodaMOSA performs significantly better, reaching a higher coverage on 173 more modules compared to MOSA and 279 more modules compared to CodexOnly.

\subsection{CodeT}

Following the examples of software artefacts presented by Microsoft developers, in 2022 is published a paper regarding CodeT~\cite{DBLP:journals/corr/abs-2207-10397}, a LLM based code generation tool.
The main purpose of CodeT is the solution of a programming problem by generating a code snippet, based on a context containing natural language.
Regardless of this not following the same pattern as previous software examples, test generation using the same LLM is part of the whole process, although all of these just have the form of a single assertion, together with a set of inputs and the desired output.

\begin{figure}[tb]
  \centering 
  \includegraphics[width=.99\textwidth]{Figures/codet.png}
  \caption{CodeT structure}\label{fig:codet}
\end{figure}

A general structure of CodeT is shown in Figure~\ref{fig:codet}, which also depicts the functioning of it by first, prompting the generation of a set of tests (right on the figure) based on the natural language context of the programming problem, that are at the same time the context for the prompting of a set of code snippets (left on the figure).
After the execution of the LLM model is finished and all the outputs are gathered, a Dual execution agreement criterion, based on the RANSAC~\cite{DBLP:journals/cacm/FischlerB81} algorithm, is used to select the best code solution.
The selection in this criterion, is done by forming consensus sets of both code solutions and test cases, by selection a valid pair \((x_\text{code}, y_\text{test})\) as a hypothetical inlier, and grouping it with all tests that are also passed by \(x_\text{code}\) in set \(S_y\) and all other code snippets that pass the same tests as \(x_\text{code}\) in set \(S_x\).
Finally, when all consensus sets \(S_x\) and \(S_y\) are formed from the different valid pairs \((x_\text{code}, y_\text{test})\), those with the highest cardinality \(f(S) = |S_x||S_y|\) have the best score for selection.

As stated before, CodeT centres around code generation rather than test case generations, which makes the research of this paper focus mainly on metrics such as pass1 or pass100 (correct rate of single solutions, or at least one for every hundred).
Nonetheless, to check the quality of the generated solutions, the authors also covered in their result analysis the code and branch coverage of the snippets generated from the different LLM models.
Almost all the models obtained an overall average coverage of \(95\%\).
\subsection{MutAP}

Mutation testing is one software testing technique, that aims to measure the ability of a test to reveal bugs.
Taking this approach in mind, Dakhel et al.\@ presented in 2023 MutAP~\cite{DBLP:journals/corr/abs-2308-16557}, another LLM based code generation tool that also introduces prompt augmentation for killing mutants in an iterative process before the actual synthesis of test cases.

\begin{figure}[tb]
  \centering 
  \includegraphics[width=.99\textwidth]{Figures/mutap.png}
  \caption{MutAP structure}\label{fig:mutap}
\end{figure}

A summary of MutAP's structure is shown in Figure~\ref{fig:mutap}, and after inputting a Python Program Under Test (PUT), it commences with the building of an initial prompt for the obtaining of an initial test case.
The first step (1), employs one of the two following learning methods; zero-shot, where the PUT is concatenated after the statement  ``Generate test case for the following code''; and few-shot, which concatenates the PUT after a sequence of pairs of a method and its respective unit test.
After the initial test is synthesized, the second step (2) consists of re-prompting it, so possible syntax errors can be corrected and unintended behaviour can be repaired.
Then, the third step (3) introduces mutations into the PUT using MutPy\footnote{https://pypi.org/project/MutPy/0.3.0/} so that the quality and effectiveness of the tests can be assessed.
At this point, the algorithm of MutAP has a conditional path depending on the number of surviving mutants; if there are none, the test cases are considered finished and generated; and in the contrary case, the prompt augmentation step is followed.
This post conditional step (4), adds information to a new prompt of the LLM, such as the already refined initial test, the statement ``The test function, \(\text{test}()\), cannot detect the fault in the
following code'', one of the mutant survivors, and a final statement ``Provide a new test case to detect the fault in prior code''.

In the experimental trial of this tool's paper, two LLM models were used, OpenAI's Codex and llama2-chat, to be run together with MutAP over the benchmarks; HumanEval, which consists of 164 human written programming problems; and Refactory, consisting of 1710 buggy submissions from students for 5 assignments of a Python programming course.
The overall results mainly cover the mutation score obtained by all the possibles methods of MutAP, obtaining numbers between \(89.13\%\) and \(93.57\%\) on the average over both models and both datasets.
\subsection{MUTester}

MUTester~\cite{DBLP:journals/corr/abs-2307-00404} is an automatic test generation research software focused on the synthesis of test cases for APIs of the many already existing Deep Learning libraries, i.e. Numpy, Scikit-learn, TensorFlow.
Its structure is presented in Figure~\ref{fig:mutester}, which describes a methodology that solely relies on heuristics regarding the mining of the documentation and knowledge repositories of the APIs in question. 

%TODO: Add numbers

\begin{figure}[tb]
  \centering 
  \includegraphics[width=.99\textwidth]{Figures/mutester.png}
  \caption{MUTester structure }\label{fig:mutester}
\end{figure}

The first half of the repository mining (1) is done by applying web scrapping to the documentation pages of different methods of the libraries, so that information regarding input type (via signature definition mining) and constraints can be gathered, which as mentioned before, is crucial for a correct test generation.
Input constraints are collected from the natural language parameter description that is usually presented in Python documentation pages, and rather than extracted using a LLM model, they are identified using by recognizing linguistic patterns.
A last piece of information that is gathered in this step are any code snippet example, so any useful input can be statically analysed.

Stack Overflow is the source of information for the second half of repository mining (2), from where the authors extracted from any related (question, accepted answer) pair the code snippets that had relevant method sequences in them.
Then (3), these method call sequences are studied using the Apriori~algorithm~\cite{agrawal1994fast} to look for intrinsic rules between the calls.

Finally, test case generation (4) is managed as a guided synthesis of a call sequence, that ends in the actual method under test, similar to what Pynguin does, but with the help of all the relationship information obtained in the previous steps.
Similarly, the input generation is conducted.

Throughout the analysis of MUTester, the API documentation and methods of the frameworks,  Scikit-learn, PyTorch, TensorFlow and CNTK were accounted for.
Regarding test coverage, the baselines for the experimentation were Pynguin and a Python implementation of Randoop, which achieved \(34.16\%\) and \(30.87\%\) line coverage in average, versus the results obtained by MUTester, that showed improvements ranging from \(15.72\%\) to \(27\%\).
A Wilcoxon signed-rank
statistical test~\cite{Rey2011} was also applied to the results, obtaining a p-value of less than \(0.05\) suggesting that MUTester outperformed both baselines in the stated datasets.
\subsection{Differential Prompting}

Detection of software failures can be acknowledged as a benefit that comes from practising software testing, and seeking this idea, Li et al.~\cite{li2023nuances} studied the implications of calling ChatGPT with prompts that bypass the insensibility of it to code nuances, so that fail inducing test cases can be generated.
Differential Prompting is the name of the proposed technique, and it divides the failure-inducing generation task in three steps:

\begin{itemize}
  \item Program intention inference, where ChatGPT is asked from a Program Under Test (PUT) to inference and explain the idea of it as a whole
  \item Program generation, where ChatGPT is again prompted to generate code snippets similar to the actual PUT, but by implicitly using the output of the last step as context
  \item Differential testing, in which a test pool of diverse inputs for the PUT is generated in order to test the references from the previous and their outputs.
  In the case that all outputs are consistent with a same set of inputs, this is marked as a ``valid test'' and compared to the output of the PUT.\@
  If they are actually the same and all the branches were already covered, then the program is marked as non-faulty.
  On the contrary, the input set is marked as a failure-inducing test case.
\end{itemize}

With these steps, Differential Prompting was put to the test against Pynguin and an out-of-the-shelf ChatGPT usage to evaluate the benchmarks QuixBugs\footnote{https://github.com/jkoppel/QuixBugs} and Codeforces\footnote{https://codeforces.com} in terms of bug presence.
In these, the success rate of Differential Prompting was \(75\%\) and \(66.7\%\) respectively, in some cases 10 times better than the baselines.

\section{SB-Based}

\subsection{EvoSuite}

With respect to Unit Test Generation, EvoSuite~\cite{DBLP:conf/qsic/FraserA11} was (and still is) a state-of-the-art test generation tool for Java, pioneer of the usage of a genetic algorithm, so the branch coverage can be optimized as a whole.
The internal representation of EvoSuite for a test suite is a set of chromosomes \(T\) as a sequence of statements \(t_i\) of length \(l_i\) and either type primitive statement, constructor statement, field statement, or method statement.
During the chromosome synthesis process, all the information regarding classes, methods and variables of the SUT are gathered via the Java Bytecode and Java Reflection\footnote{https://docs.oracle.com/javase/8/docs/technotes/guides/reflection/index.html}.
These sequences can be crossovered as a random sequential combination of the two chromosomes, or mutated with the insertion, remotion or change of single statements.
In terms of the optimization, the fitness function of a test suite $T$ is given by the equation
\[ \text{fitness}(T) = |M| - |M_T| + \sum_{b_k \in B} d(b_k, T) \]
where \(M\) is the total of methods to test, \(M_T\) is the number of methods that were actually executed, \(b_k\) is a branch in the branch set \(B\), and \(d(b, T)\) is the branch distance of branch \(b\) over the set \(T\).
Another limitation to the suites \(T\), are \(N\) as the maximum number of chromosomes (\(T = \{t_1, \dots , t_n\}\)) and \(L\) as maximum length of the chromosomes (\(l_i < L\)).
For the experiment phase, a single branch approach was created by the authors, in which every branch \(b_t\) is seen as a single coverage goal to be met by the test cases.
Then, the comparison between the single branch method and the whole suite optimization is done with the generation of test cases for five open source projects, accumulating a total of 1308 classes from which 727 were public.
After the execution of test cases was complete and statistical tests were applied, it could be stated that the results show EvoSuite has a better branch coverage and creates smaller tests suites than the single objective method.
\subsection{NxtUnit}

\begin{figure}[bt]
  \centering
  \includegraphics[width=.99\textwidth]{Figures/nxtunit2.png}
  \caption{NxtUnit structure}\label{fig:nxt}
\end{figure}

Having EvoSuite, Randoop and Pynguin as precedents, in 2023 developers from ByteDance (technology Chinese company) presented NxtUnit~\cite{DBLP:conf/ease/WangMCGSP23}, an automatic randomized test generation tool for the programming language Go.
Similar to other state-of-the-art tools, the structure of NxtUnit (see Figure~\ref{fig:nxt}) is composed by an initial source code as input, that is translated into Static Single Assignment form as a way to get a graph with call dependencies between functions.
After this preprocessing, NxtUnit proceeds to synthesize from the call graph a code template that is filled with appropriate randomly generated parameters and ran during the tool's execution.
All the information for the template synthesis and posterior parameter changes are done thanks to the information provided by Go Reflect\footnote{https://pkg.go.dev/reflect}.
During the code running phase, NxtUnit focuses in three main tasks: mutation of the parameter inputs at runtime, assertion generation for future regression testing, and test case selection based on code coverage.
Once the tool comes to a halt, the best test cases are exported into a test suite.
In the experiment proposed by the authors, NxtUnit was executed on 500 private ByteDance repositories and 13 out of the 100 highly-rated public repositories of Go, both sets that already contain tests.
The main objective of this setup, was to review the code coverage increase while adding NxtUnit tests to the already existing ones, and it produced an average increase of \(5.51\%\) for the public repositories and \(17.26\%\) for the private ones.
\subsection{JSEFT}

As an example for automatic test generation in dynamic languages others than Python, JSEFT is presented in 2015 by Mirshokraie et al.~\cite{DBLP:conf/icst/Mirshokraie0P15} as a tool for generating not only unit tests, but also event-driven tests for web applications written in JavaScript.
The general structure of JSEFT is presented in Figure~\ref{fig:jseft} and is composed of three main steps that process the data given by the HTML Document Object Model (DOM):
\begin{itemize}
  \item In (1) the web app is dynamically analysed in order to get all the possible states of the program and, in consequence, construct a state flow graph (SFG)
  \item Then, in step (2), event sequences are extracted from the SFG for the creation of event-driven tests, which are run in the instrumentalized version of the web app.
  With these executions, JSEFT tries to discover DOM states and entry and exit state points for the creation of function-level unit tests
  \item Finally, the third step (3) mutates the web app at code and DOM level, so functional oracles can be created from the difference between the two states of the original and mutated tests versions
\end{itemize}

\begin{figure}[tb]
  \centering 
  \includegraphics[width=.99\textwidth]{Figures/jseft2.png}
  \caption{JSEFT structure}\label{fig:jseft}
\end{figure}

For the export of event-driven and function-level tests, the frameworks Selenium\footnote{https://www.selenium.dev} and QUnit\footnote{https://qunitjs.com} are used respectively.

The experiment section of the author's paper, shows the study of JSEFT's usage on 13 JavaScript web applications so the line coverage and usefulness of the assertions at the time of finding regression faults.
For the specified metrics, the tool got a \(68.4\%\) of line coverage and a \(100\%\) precision in fault detection.

\chapter{Conclusions}~\label{chap:conclusions}

Comments on
\begin{itemize}
  \item Current state of Python as how its developers use it.
  \item Effectiveness of the GBOS extension in this current state.
  \item Summary of the research questions.
  \item Other questions? (ask Stephan).
  \item Comments on the future of Pynguin and Python, considering the state of the art and the study's results.
\end{itemize}

\section{Future Work}

Comments on
\begin{itemize}
  \item Improvement of the static analysis of the Data Dependencies, for an improved OCG construction.
  \item Improvements in terms of optimization over the whole GBOS process.
  \item Link some comment about the state of the art and the usage of LLMs.
\end{itemize}

\backmatter{}

\printbibliography{}

\end{document}
